%%%%%%%%%%%%%%%%%%%%%%%%%%%%%%%%%%%%%%%%%%%%%%%%%%%%%%%%%%%%%%%%%%%%%%%%%%%%
\documentclass[12pt,leqno]{amsart}
\usepackage[width=15.00cm, height=20.00cm, margin=3.00cm, headsep=0.70cm]{geometry}
\setlength{\parskip}{0.6\baselineskip}

\usepackage{palatino} % For palatino style, may be used/removed. 
\usepackage{amsmath,amssymb,amsxtra,color,calligra,mathrsfs,comment,url}
%\let\circledS\undefined % here - PS
%\usepackage[bitstream-charter, cal=cmcal]{mathdesign} 

\usepackage{xcolor} % Required for specifying custom colours 
\colorlet{mdtRed}{red!50!black}
\colorlet{dblue}{blue!50!black}
\usepackage[colorlinks, pagebackref=true]{hyperref}
\hypersetup{linkcolor=dblue,citecolor=dblue,filecolor=dullmagenta,urlcolor=mdtRed}
\renewcommand*{\backref}[1]{}
\renewcommand*{\backrefalt}[4]{[{%
		\ifcase #1 Not cited.%
		\or $\uparrow$~#2.%
		\else $\uparrow$~#2.%
		\fi%
	}]}
\usepackage[all]{xy}
\usepackage{tikz,tikz-cd,tkz-graph,enumerate}
\usetikzlibrary{matrix,arrows,decorations.pathmorphing}
\usepackage{dsfont}


%%%%%%%%%%%%%%% Some shortcuts %%%%%%%%%%%%%%%%%
\DeclareMathOperator{\Hom}{\textnormal{Hom}}
\DeclareMathOperator{\sHom}{\mathcal{H}\!\textit{om}}
\DeclareMathOperator{\End}{\textnormal{End}}
\DeclareMathOperator{\sEnd}{\mathcal{E}\!\textit{nd}}
\DeclareMathOperator{\Iso}{\textnormal{Iso}}
\DeclareMathOperator{\Mor}{\textnormal{Mor}}
\DeclareMathOperator{\gr}{\textnormal{gr}}
\DeclareMathOperator{\Flag}{\textnormal{Flag}}
\DeclareMathOperator{\rk}{\mathrm{rk}}
\DeclareMathOperator{\Id}{\textnormal{Id}}

\DeclareMathOperator{\Set}{\textnormal{Set}}
\DeclareMathOperator{\ad}{\textnormal{ad}}
\DeclareMathOperator{\Ad}{\textnormal{Ad}}
\DeclareMathOperator{\Aut}{\textnormal{Aut}}
\DeclareMathOperator{\aut}{\textnormal{aut}}
\DeclareMathOperator{\Lie}{\textnormal{Lie}}
\DeclareMathOperator{\GL}{\textnormal{GL}}
\DeclareMathOperator{\SL}{\textnormal{SL}}
\DeclareMathOperator{\PGL}{\textnormal{PGL}}
\DeclareMathOperator{\Spec}{\textnormal{Spec}}
\DeclareMathOperator{\Pic}{\textnormal{Pic}}
\DeclareMathOperator{\Bun}{\mathcal{B}\textit{un}}
\DeclareMathOperator{\M}{\textnormal{M}}
\DeclareMathOperator{\dR}{\textnormal{dR}}

\DeclareMathOperator{\Higgs}{\mathcal{H}\textit{iggs}}
\DeclareMathOperator{\et}{{\textnormal{\'et}}}
\DeclareMathOperator{\Sch}{\textnormal{Sch}\!}
\DeclareMathOperator{\ob}{\textnormal{Ob}}

\DeclareMathOperator{\Tors}{\mathcal{T}\textit{ors}}
\DeclareMathOperator{\Rep}{\mathcal{R}\textit{ep}}
\DeclareMathOperator{\Vect}{\mathcal{V}\textit{ect}}
\DeclareMathOperator{\Fun}{\mathcal{F}\textit{un}}
\DeclareMathOperator{\Mod}{\mathrm{Mod}}

\DeclareMathOperator{\QCoh}{\mathfrak{QCoh}}
\DeclareMathOperator{\Coh}{\mathfrak{Coh}}
\DeclareMathOperator{\FM}{\mathrm{FM}}
\DeclareMathOperator{\Hilb}{\mathcal{H}\!\textit{ilb}}
\DeclareMathOperator{\Supp}{\mathrm{Supp}}

\DeclareMathOperator{\sExt}{\mathcal{E}\!\textit{xt}}
\DeclareMathOperator{\Ext}{\textnormal{Ext}}
\DeclareMathOperator{\codim}{\mathrm{codim}}
\DeclareMathOperator{\hd}{\mathrm{dh}}
\DeclareMathOperator{\depth}{\mathrm{depth}}
\DeclareMathOperator{\HN}{\mathrm{HN}}
\DeclareMathOperator{\Kom}{\mathcal{K}\!\textit{om}}
\DeclareMathOperator{\Coim}{\mathrm{Coim}}

\DeclareMathOperator{\Coker}{\mathrm{Coker}}
\DeclareMathOperator{\CH}{\mathrm{CH}}
\DeclareMathOperator{\ch}{\mathrm{ch}}

\DeclareMathOperator{\supp}{\mathrm{Supp}}
\DeclareMathOperator{\Ass}{\mathrm{Ass}}
\DeclareMathOperator{\nf}{{\rm nf}}
\DeclareMathOperator{\sGr}{\mathcal{G}\textit{r}}
\DeclareMathOperator{\Gr}{\textnormal{Gr}}
\DeclareMathOperator{\Hnf}{\textrm{H-nf}}

\renewcommand{\ker}{\mathrm{Ker}}
\renewcommand{\Im}{\mathrm{Im}}
\renewcommand{\Re}{\mathrm{Re}}


%%% two right arrows. 
\usepackage{extpfeil}
\newextarrow{\xbigtoto}{{20}{20}{20}{20}}
{\bigRelbar\bigRelbar{\bigtwoarrowsleft\rightarrow\rightarrow}}

\newcommand{\mf}[1]{\mathfrak{#1}}
\newcommand{\mc}[1]{\mathcal{#1}}
\newcommand{\ms}[1]{\mathscr{#1}}
\newcommand{\bb}[1]{\mathbb{#1}}
\newcommand{\dv}{{\vee\vee}}
\newcommand{\ul}[1]{\underline{#1}}
\newcommand{\dual}{^\vee}
\newcommand{\ds}[1]{\mathds{#1}}

\numberwithin{equation}{section}

\newtheorem{theorem}[equation]{Theorem}
\newtheorem{corollary}[equation]{Corollary}
\newtheorem{lemma}[equation]{Lemma}
\newtheorem{proposition}[equation]{Proposition}
\newtheorem{notation}[equation]{Notation}
\newtheorem{conjecture}[equation]{Conjecture}

\newtheorem{fact}{Fact}
\newtheorem*{theorem-nonum}{Theorem}

\theoremstyle{definition}
\newtheorem{definition}[equation]{Definition}
\newtheorem{remark}[equation]{Remark}
\newtheorem{example}[equation]{Example}
\newtheorem{question}[equation]{Question}
\newtheorem{exercise}[equation]{Exercise}


%%%%%%%% Symbols/letters/numbers for \thanks{} command: 
\makeatletter
\newcommand\thankssymb[1]{\textsuperscript{\@fnsymbol{#1}}}
\newcommand\thanksletter[1]{\lowercase{\textsuperscript{\@alph{#1}}}}
\newcommand\thanksnum[1]{\textsuperscript{#1}}
\makeatother

\newcommand{\corauth}[2][]{\thanks{#2Corresponding author}}

%%%%% To use AMS 2020 MSC: 
\makeatletter
\@namedef{subjclassname@2020}{%
	\textup{2020} Mathematics Subject Classification}
\makeatother 

%%%%%% Table of contents' depth level: 
\setcounter{tocdepth}{2}
\setcounter{secnumdepth}{2}

\begin{document}
\date{\today}

\baselineskip=16pt
\section{Model Answer}
\noindent
{\bf Question.} 
Use Plancherel's identity to find a relation between the solution and the initial data 
in terms of mean square norm called \(L^2\) norm of the Cauchy problem: 
$$u_t+u_{xxxx} +2u=0,\;-\infty < x < \infty,\, t>0$$ 
with initial data \(u(x,0)=u_0(x),\;\; -\infty < x < \infty\); and also find 
$\lim\limits_{t\to\infty} \int_{-\infty}^{\infty} |u(x,t)|^2\;dx$.

\begin{proof}[Model Answer:]
We have given 
\begin{equation}\label{eqn1}
u_t + u_{xxxx} + 2u = 0,\ -\infty < x < \infty,\ t > 0, 
\end{equation}
with initial data $u(x, 0) = u_0(x),\ -\infty < x < \infty$. 
Taking Fourier transform with respect to $x$, from equation \eqref{eqn1}, we get 
\begin{align}
&\ \widehat{u}_t(w, t) + (iw)^4\widehat{u} + 2\widehat{u} = 0 \nonumber \\
\Longrightarrow &\ \widehat{u}_t(w, t) + (w^4+2)\widehat{u} = 0 \nonumber \\ 
\Longrightarrow &\ \widehat{u}(w, t) = \widehat{u}_0(w) e^{-t(w^4+2)} \label{eqn2}
\end{align}
Using Plancherel's identity and \eqref{eqn1}, we get 
\begin{align*}
\int_{-\infty}^\infty |u(x, t)|^2 dx & = \int_{-\infty}^\infty |\widehat{u}(w, t)|^2 dw 
= e^{-4t} \int_{-\infty}^\infty e^{-2tw^4}\cdot |\widehat{u}_0(w)|^2dw. %\text{ using \eqref{eqn2}.} \nonumber 
\end{align*}
Since for each real number $t > 0$, we have $e^{-2tw^4} \leq 1, \forall\, w \in \bb R$, from above equation we have 
\begin{equation}\label{eqn3}
	\int_{-\infty}^\infty |u(x, t)|^2 dx \leq e^{-4t} \int_{-\infty}^\infty |\widehat{u}_0(w)|^2 dw. 
\end{equation}
Since $e^{-4t} \leq 1$, for all $t > 0$, again using Plancherel's identity, 
we find that the relation between the solution and the initial data is 
$$\int_{-\infty}^{\infty} |u(x,t)|^2 dx 
\leq \int_{-\infty}^{\infty} |\widehat{u_0}(w)|^2 dw 
= \int_{-\infty}^\infty |u_0(x)|^2 dx,\ \ \forall\ t > 0.$$ 
Taking limit as $t \to \infty$ in \eqref{eqn3}, we get 
\begin{align}
	& \ 0 \leq \lim\limits_{t\to\infty} \int_{-\infty}^{\infty} |u(x, t)|^2 dx 
	\leq \lim\limits_{t\to\infty} e^{-4t}\int_{-\infty}^{\infty} |\widehat{u_0}(w)| dw = 0. \nonumber \\ 
	\Longrightarrow & \ \ \lim\limits_{t\to\infty} \int_{-\infty}^{\infty} |u(x, t)|^2 dx  = 0. \nonumber
\end{align}
\end{proof}


\date{\today}




\end{document}